% Options for packages loaded elsewhere
\PassOptionsToPackage{unicode}{hyperref}
\PassOptionsToPackage{hyphens}{url}
%
\documentclass[
]{book}
\usepackage{amsmath,amssymb}
\usepackage{iftex}
\ifPDFTeX
  \usepackage[T1]{fontenc}
  \usepackage[utf8]{inputenc}
  \usepackage{textcomp} % provide euro and other symbols
\else % if luatex or xetex
  \usepackage{unicode-math} % this also loads fontspec
  \defaultfontfeatures{Scale=MatchLowercase}
  \defaultfontfeatures[\rmfamily]{Ligatures=TeX,Scale=1}
\fi
\usepackage{lmodern}
\ifPDFTeX\else
  % xetex/luatex font selection
\fi
% Use upquote if available, for straight quotes in verbatim environments
\IfFileExists{upquote.sty}{\usepackage{upquote}}{}
\IfFileExists{microtype.sty}{% use microtype if available
  \usepackage[]{microtype}
  \UseMicrotypeSet[protrusion]{basicmath} % disable protrusion for tt fonts
}{}
\makeatletter
\@ifundefined{KOMAClassName}{% if non-KOMA class
  \IfFileExists{parskip.sty}{%
    \usepackage{parskip}
  }{% else
    \setlength{\parindent}{0pt}
    \setlength{\parskip}{6pt plus 2pt minus 1pt}}
}{% if KOMA class
  \KOMAoptions{parskip=half}}
\makeatother
\usepackage{xcolor}
\usepackage{longtable,booktabs,array}
\usepackage{calc} % for calculating minipage widths
% Correct order of tables after \paragraph or \subparagraph
\usepackage{etoolbox}
\makeatletter
\patchcmd\longtable{\par}{\if@noskipsec\mbox{}\fi\par}{}{}
\makeatother
% Allow footnotes in longtable head/foot
\IfFileExists{footnotehyper.sty}{\usepackage{footnotehyper}}{\usepackage{footnote}}
\makesavenoteenv{longtable}
\usepackage{graphicx}
\makeatletter
\def\maxwidth{\ifdim\Gin@nat@width>\linewidth\linewidth\else\Gin@nat@width\fi}
\def\maxheight{\ifdim\Gin@nat@height>\textheight\textheight\else\Gin@nat@height\fi}
\makeatother
% Scale images if necessary, so that they will not overflow the page
% margins by default, and it is still possible to overwrite the defaults
% using explicit options in \includegraphics[width, height, ...]{}
\setkeys{Gin}{width=\maxwidth,height=\maxheight,keepaspectratio}
% Set default figure placement to htbp
\makeatletter
\def\fps@figure{htbp}
\makeatother
\setlength{\emergencystretch}{3em} % prevent overfull lines
\providecommand{\tightlist}{%
  \setlength{\itemsep}{0pt}\setlength{\parskip}{0pt}}
\setcounter{secnumdepth}{5}
\usepackage{booktabs}
\usepackage{amsthm}
\makeatletter
\def\thm@space@setup{%
  \thm@preskip=8pt plus 2pt minus 4pt
  \thm@postskip=\thm@preskip
}
\makeatother
\ifLuaTeX
  \usepackage{selnolig}  % disable illegal ligatures
\fi
\usepackage[]{natbib}
\bibliographystyle{apalike}
\IfFileExists{bookmark.sty}{\usepackage{bookmark}}{\usepackage{hyperref}}
\IfFileExists{xurl.sty}{\usepackage{xurl}}{} % add URL line breaks if available
\urlstyle{same}
\hypersetup{
  pdftitle={The Historical Context of Confucianism and Neo-Confucianism},
  pdfauthor={Yuxuan Su},
  hidelinks,
  pdfcreator={LaTeX via pandoc}}

\title{The Historical Context of Confucianism and Neo-Confucianism}
\author{Yuxuan Su}
\date{2024-05-20}

\begin{document}
\maketitle

{
\setcounter{tocdepth}{1}
\tableofcontents
}
Brief Research Conclusion

\hypertarget{brief-conclusion}{%
\chapter{Brief Conclusion}\label{brief-conclusion}}

儒家思想是先秦诸子百家思想的重要分支,也是中国传统文化的重要组成部分。儒家思想与中华文明伴生伴行,深刻影响着当今中国的国家治理、社会观念与思想文化,也对东亚文化圈的形成起到了直接的推动作用。

商周时期,儒家思想舒展萌芽,``儒''的观念逐渐在齐鲁大地上传播。作为儒家思想的创始人,孔子将``仁礼忠恕''、``为政以德''作为修身治国的核心道义。战国时期,孟子、荀子等人在诸侯征伐与百家争鸣中进一步发展了儒家思想,提出了``民贵君轻''、``礼乐化人''的著名论断,儒家思想在适应社会需求中进一步体系化和完整化。战国后期,儒学在诸子百家中蔚然大宗。``焚书坑儒''后,儒家思想在秦朝趋于沉寂。

西汉初年,统一的中央集权国家需要新的政治思想来扭转内外松弛的局势,董仲舒提出``独尊儒术''、``天人合一''的主张,并以此维护封建王朝的统治秩序。随着儒家学说成为国家授官任爵的主要标准,儒学逐渐走向制度化和宗教化,成为意识形态的主流。魏晋至隋唐是儒学的危机时期,儒、道、佛的学说在纷争中融合,``三百年间无大儒'',这一时期儒学发展的脚步趋缓。

唐末宋初,长期以来的社会失序呼唤着儒家思想的新陈代谢,佛道两家的日益坐大对儒学提出了严峻的挑战。这一背景下,重构自身的价值体系,论证传统伦理规范成为儒家思想进一步发展必须担负的重任,新儒学,也即宋明理学应运而生。程朱理学将孔孟之政治伦理思想升华至追问世界终极存在之本体论高度,新儒学充分发挥儒学道德修养方面的社会功能,促成儒学质的飞跃。

明清时期是儒家思想的继承与批判时期。随着资本主义萌芽的产生与中央集权的空前强化,黄宗羲、顾炎武、王夫之等人提出了反对君主专制的早期民主思想和工商皆本的经济主张,在批判中发展着儒家思想。近代以来,``尊孔复古''与``全面西化''的思想潮流相互激荡,学者们开始思考关于孔子评价和中国传统文化的批判与继承。

新中国成立以来,共产革命的左倾思潮破坏了儒家思想的土壤,儒家在中国大陆与海外的发展均陷入了沉寂。改革开放以来,儒学问题探讨的深度与广度双向拓展,现代``新儒学''在探索中发展,但总的来说,并没有形成统一而权威的当代儒学体系。

Confucianism is an important branch of the pre-Qin Hundred Schools of Thought and a significant component of traditional Chinese culture. It has evolved alongside Chinese civilization, profoundly influencing contemporary China's governance, social concepts, and cultural thought, and has played a direct role in the formation of the East Asian cultural sphere.

During the Shang and Zhou periods, Confucian thought began to sprout, with the concept of ``Ru'' (Confucian scholars) gradually spreading across the land of Qi and Lu. As the founder of Confucianism, Confucius emphasized ``benevolence, propriety, loyalty, and forgiveness'' and ``governing with virtue'' as the core principles for personal conduct and state governance. During the Warring States period, thinkers such as Mencius and Xunzi further developed Confucianism amid the conflicts of feudal lords and the Hundred Schools of Thought, proposing famous doctrines like ``the people are of paramount importance, while the ruler is of lesser importance'' and ``using rites and music to civilize the people.'' Confucianism became more systematized and complete in response to societal needs. By the late Warring States period, Confucianism had become a dominant school of thought among the Hundred Schools. However, it fell into decline during the Qin Dynasty's ``burning of books and burying of scholars.''

At the beginning of the Western Han Dynasty, the unified centralized state needed a new political ideology to stabilize internal and external tensions. Dong Zhongshu proposed the ideas of ``suppressing the hundred schools and honoring only Confucianism'' and ``the unity of heaven and humanity'' to maintain the feudal order. As Confucian doctrines became the main criteria for official appointments, Confucianism gradually became institutionalized and religious, emerging as the dominant ideology. From the Wei and Jin periods to the Sui and Tang dynasties, Confucianism faced a crisis as Confucian, Taoist, and Buddhist teachings blended and conflicted. This period saw a slowdown in Confucian development, with ``no great Confucians in three hundred years.''

By the late Tang and early Song periods, long-term social disorder called for a renewal of Confucian thought. The growing influence of Buddhism and Taoism posed significant challenges to Confucianism. In this context, Confucianism needed to reconstruct its value system and justify traditional ethical norms, leading to the emergence of Neo-Confucianism, also known as Song-Ming Rationalism. The Cheng-Zhu school elevated Confucius and Mencius' political and ethical ideas to the level of metaphysical inquiry into the ultimate nature of existence. Neo-Confucianism significantly enhanced the social function of Confucian moral cultivation, marking a qualitative leap for Confucianism.

The Ming and Qing periods were times of both inheritance and critique for Confucian thought. With the rise of capitalist tendencies and the unprecedented strengthening of centralized authority, thinkers like Huang Zongxi, Gu Yanwu, and Wang Fuzhi introduced early democratic ideas opposing monarchical despotism and advocated for economic principles valuing both agriculture and commerce. These critiques helped develop Confucian thought further. In modern times, the intellectual currents of ``revering Confucius and restoring antiquity'' and ``comprehensive Westernization'' clashed, prompting scholars to reconsider Confucius' legacy and the critique and inheritance of traditional Chinese culture.

Since the establishment of the People's Republic of China, leftist revolutionary currents have undermined the foundations of Confucian thought, leading to its dormancy both on the mainland and overseas. Since the reform and opening-up era, the exploration of Confucian issues has expanded in both depth and breadth. Modern ``New Confucianism'' continues to develop through exploration, but overall, a unified and authoritative contemporary Confucian system has yet to emerge.

\hypertarget{emerging-stage}{%
\chapter{Emerging Stage}\label{emerging-stage}}

The academic community generally believes that the concept of ``Ru''(儒) originated during China's Shang and Zhou periods. The term ``Confucianism'' first appeared in the ``Records of the Grand Historian''(史记) : ``The people of Lu all follow Confucian teachings, and Zhu Jia is known for his chivalry.''(``鲁人皆以儒教,而朱家用侠闻。'') This indicates that the State of Lu during the pre-Qin period was a place where Confucianism flourished. Confucius(孔子) systematically summarized the concept of ``Ru'' into a complete ideological system, thus he is regarded as the founder of Confucian thought. The Pre-Qin period was a crucial time for the formation and development of Confucianism. Confucian thought primarily emerged and gradually expanded during the Spring and Autumn and Warring States periods. Key figures representing Confucianism during this time include Confucius, Mencius, and Xunzi.

\hypertarget{overview}{%
\section{Overview}\label{overview}}

\textbf{Time}

7th BC to 2th BC

\textbf{Representative Figures}

Confucius(孔子), Mencius(孟子), Xunzi(荀子)

\hypertarget{ideological-tenets}{%
\section{Ideological Tenets}\label{ideological-tenets}}

\hypertarget{confucius}{%
\subsection{Confucius}\label{confucius}}

As mentioned earlier, Confucius was the most significant Confucian thinker during the Pre-Qin period, and his ideological tenets are mainly recorded in the ``Analects.'' Confucius advocated for benevolence, righteousness, propriety, wisdom, and trustworthiness, emphasizing the importance of harmonious relationships between individuals and the social order. He believed that through education and self-cultivation, everyone could become a ``gentleman,'' characterized by moral excellence. Confucius also valued ritual and music systems, considering them as the foundation for maintaining social order and stability.

\hypertarget{mencius}{%
\subsection{Mencius}\label{mencius}}

Mencius, following Confucius, was another important Confucian thinker, and his ideas are primarily documented in the ``Mencius.'' Mencius further developed Confucius' notion of benevolent governance, proposing the viewpoint that ``the people are the most important, followed by the altars of the land and grain, and the ruler is the lightest.'' He emphasized that rulers must implement benevolent governance, caring for the people, in order to legitimately exercise authority. Mencius also proposed the theory of innate goodness, believing that human nature is inherently good, and that every person possesses an innate drive towards goodness.

\hypertarget{xunzi}{%
\subsection{Xunzi}\label{xunzi}}

Xunzi, another important figure of Confucianism during the Warring States period, is documented primarily in the book ``Xunzi.'' In contrast to Mencius' theory of innate goodness, Xunzi proposed the theory of innate evil, suggesting that human nature is inherently wicked and requires the constraints of education and rituals to regulate human behavior. He emphasized the concept of ``transforming nature through training,'' which means changing human nature through learning and self-cultivation to achieve social harmony. Xunzi also placed great importance on rituals and laws, considering them fundamental to maintaining social order.

\hypertarget{historical-status}{%
\section{Historical Status}\label{historical-status}}

Just as the Qing dynasty scholar Wang Guowei(王国维) remarked, ``The most dramatic changes in Chinese politics and culture occurred during the transition from the Shang to the Zhou dynasty.'' Confucianism continuously enriched and perfected during the Pre-Qin period, gradually forming a theoretical system guiding political governance and social life. This had a profound impact on Chinese history and culture. The core concepts of Confucianism during this period, such as benevolent governance, rituals and music, innate goodness or evil, laid the foundation for the significant position of Confucian thought in the intellectual history of China.

\hypertarget{institutionalized-and-ritualized-confucianism}{%
\chapter{Institutionalized and Ritualized Confucianism}\label{institutionalized-and-ritualized-confucianism}}

\hypertarget{overview-1}{%
\section{Overview}\label{overview-1}}

\textbf{Time}

2th BC to 8th AD

\textbf{Representative Figures}

Dong Zhongshu(董仲舒), Han Yu(韩愈), Ji Kang(嵇康), Ruan Ji(阮籍)

\hypertarget{ideological-tenets-1}{%
\section{Ideological Tenets}\label{ideological-tenets-1}}

\hypertarget{han-dynasty-honoring-only-confucianism}{%
\subsection{Han Dynasty: honoring only Confucianism}\label{han-dynasty-honoring-only-confucianism}}

Dong Zhongshu (179 B.C. -- 104 B.C.) was one of the key figures representing Confucianism during the Han dynasty. He proposed the cosmological concept of the unity of Heaven and humanity, advocating for a close connection and mutual influence between humans and all things in the universe. He emphasized the idea of Tianming (the Mandate of Heaven), believing that human destiny is constrained and arranged by the will of Heaven. He advocated for rulers to align themselves with the Mandate of Heaven, governing the state based on it, implementing benevolent governance to maintain social stability and the well-being of the people. Dong Zhongshu also advocated for the divine bestowment of royal authority, believing that the authority of rulers is bestowed by Heaven. Rulers, as individuals appointed by Heaven, should exercise political power based on the Mandate of Heaven to uphold social order and stability.

\hypertarget{wei-jin-to-sui-and-tang-institutionalized-confucianism}{%
\subsection{Wei, Jin to Sui and Tang: Institutionalized Confucianism}\label{wei-jin-to-sui-and-tang-institutionalized-confucianism}}

After the Eastern Jin and the Southern and Northern Dynasties, and extending into the Sui and Tang periods, the influence of Buddhist thought surpassed that of Neo-Daoism (Xuanxue), playing a significant role in the intellectual cultivation of the scholar-official class. Therefore, during the approximately seven hundred years from the Wei, Jin, and Southern and Northern Dynasties to the end of the Five Dynasties in the late Tang, Confucianism continued to function primarily in its institutionalized political aspects, upheld by the ruling class.

Overall, from the Wei and Jin periods to the Sui and Tang periods, the ritualization of Confucianism was closely related to its institutionalization in politics, with both processes occurring simultaneously and interdependently. The former served to establish and consolidate the latter. The formation and strengthening of Confucianism's function in the social and political spheres consequently weakened its role as a framework for general ethical and moral cultivation and as a vehicle for political ideals.

\hypertarget{historical-status-1}{%
\section{Historical Status}\label{historical-status-1}}

During the Han dynasty, the political influence of Confucianism continued to expand, becoming an essential theoretical foundation for rulers to formulate policies and govern the state. The Confucian principle of benevolent governance received attention from rulers, and Confucian scholars became significant targets for the selection and training of officials. Confucian classics also became the main content of official education. Confucian ideology permeated various levels of society, influencing people's behavior and moral concepts. The Confucian concepts of rituals and music were widely applied in social life, and Confucian educational ideas also affected both family education and societal norms.

\hypertarget{neo-confucianism}{%
\chapter{Neo-Confucianism}\label{neo-confucianism}}

\hypertarget{overview-2}{%
\section{Overview}\label{overview-2}}

\textbf{Time}

9th century to the 11th century AD

\textbf{Representative Figures}

Wang Yangming(王阳明), Zhu Xi(朱熹), Cheng Hao(程颢)

\hypertarget{causes-of-emergence}{%
\section{Causes of Emergence}\label{causes-of-emergence}}

\hypertarget{long-term-social-disorder}{%
\subsection{Long-term social disorder}\label{long-term-social-disorder}}

From the late 8th to the early 9th century, the Tang Dynasty experienced the turbulence of the An Lushan Rebellion. The once-powerful Tang Dynasty began to decline, and powerful regional military governors formed a situation where local military strength surpassed the central authority. This loss of political authority led to confusion in the social and intellectual spheres. Traditional Confucian ethical norms seemed to lose their proactive control over social thought and the unquestionable authority they once held. This state of intellectual disorder persisted until the early Northern Song Dynasty. In the 1060s, after the campaigns against the Southern Tang and Northern Han, the Northern Song government largely restored unity to China. However, after a long period of disorder and chaos, it was challenging for social thought to immediately return to a situation where Confucianism dominated. This provided an opportunity for the revival of Confucianism.

\hypertarget{impact-of-buddhism-and-taoism}{%
\subsection{Impact of Buddhism and Taoism}\label{impact-of-buddhism-and-taoism}}

Since the Wei, Jin, Southern and Northern Dynasties, Buddhism and Taoism have been eroding the territory of mainstream Confucianism through conflict, coordination, and adaptation processes. The Sinicization of Buddhism accelerated after the Southern and Northern Dynasties, and by the time of the Sixth Patriarch Huineng and the emergence of the sudden enlightenment Chan (Zen) sect, it had firmly established itself in Chinese intellectual circles. Although Sinicized Buddhism absorbed a large amount of Confucian thought and terminology to cater to the Chinese people's mindset, its goal of transcendence remained fundamentally different from the Confucian emphasis on worldly affairs.

Meanwhile, indigenous Taoism, rooted in Chinese soil, began to rapidly develop by incorporating classical texts and borrowing some terminology from Buddhism. At one point during the Tang Dynasty, it was even honored as the state religion. The widespread popularity of Buddhism and Taoism, coupled with the central government's repeated overtures to them, undoubtedly sounded alarm bells for conservative Confucian intellectuals: Confucianism had reached a point where it required significant transformation to survive, or else it would struggle to maintain relevance.

\hypertarget{the-demand-to-rebuild-moral-authority}{%
\subsection{The demand to rebuild moral authority}\label{the-demand-to-rebuild-moral-authority}}

Since the Han Dynasty, despite enduring numerous trials and undergoing several transformations, Confucianism's position as the paramount social and ethical ideology has only solidified. Confucian thought has long been the mainstream ideology in society and the foundational ideology for state and social authority. The Confucian system of values, ethical concepts, and the formalized ritualistic system has always been unquestionable. However, during the Tang and Song periods, due to intellectual chaos and the popularity of Buddhism and Taoism, people began to doubt traditional Confucianism and its lack of systematic ontological support.

This skepticism posed a challenge to traditional Confucianism, as it lacked a methodological approach to address the doubts arising from the absence of systematic ontological arguments. The problem facing intellectuals of the Song and Ming dynasties was clear: How to dispel people's doubts and restore the authority of Confucian ethical and moral thought? One of the central tasks in addressing this issue was to systematically justify traditional ethical norms and concepts.

\hypertarget{historical-status-2}{%
\section{Historical Status}\label{historical-status-2}}

Neo-Confucianism of the Song and Ming dynasties represents the most speculative, comprehensive, and modern philosophical trend in the history of Chinese philosophy. The rise and development of Neo-Confucianism during the Song and Ming dynasties significantly restored Confucianism's social function in the realm of ethics and personal cultivation. This restoration complemented and reinforced the political institutional aspect of Confucianism, thereby enhancing its dual role in society's governance and education. After the Song and Ming periods, this harmonization of Confucianism's two levels of social functions---ethical cultivation and political institution---often led to the entanglement of issues originally belonging to the ethical domain with those of the political domain, making it difficult to distinguish between the two.

Moreover, since the ethical cultivation aspect directly served the political institutional aspect, norms that were originally based on principles of self-awareness and voluntary adherence often became enforced rules. These rules, governed by concepts such as ``Heavenly principles'' and ``conscience,'' were sometimes more stringent than explicitly stated laws. Consequently, the regulations intended to guide moral behavior could be more severe than formal legal codes.Since modern times, especially since the May Fourth Movement, Confucianism has faced intense criticism, primarily targeting the Neo-Confucianism (Xingli) of this period.

\hypertarget{new-confucianism}{%
\chapter{New Confucianism}\label{new-confucianism}}

\hypertarget{overview-3}{%
\section{Overview}\label{overview-3}}

\textbf{Time}

17th century to the 21th century AD

\textbf{Representative Figures}

Gu Yanwu(顾炎武), Huang Zongxi(黄宗羲), Liang Shuming(梁漱溟), Mou Zongsan(牟宗三)

\hypertarget{late-qing-and-republican-periods}{%
\section{Late Qing and Republican periods}\label{late-qing-and-republican-periods}}

The Confucianism in the Qing Dynasty was deeply influenced by politics, especially after the reign of Emperor Qianlong. The court proclaimed its respect for Song Dynasty learning (though it actually referred to Neo-Confucianism, while the study of the mind, known as Xinxue, was considered heterodox during the Qing Dynasty). Scholars in society no longer practiced Daoism like their predecessors in the Song and Ming dynasties; instead, they focused mainly on textual analysis, claiming to revive the tradition of Han Dynasty studies. During the late Qing and Republican periods, warlord conflicts and cultural decline resulted in a slow development of Confucian thought.

\hypertarget{confucianism-and-cpc}{%
\section{Confucianism and CPC}\label{confucianism-and-cpc}}

From the establishment of the People's Republic of China to the end of the Cultural Revolution, this period marked a stage of dormancy for Confucian thought. During this time, research on Neo-Confucianism not only stagnated on the Chinese mainland but also saw little innovation in Hong Kong, Taiwan, and overseas. The stage of recovery and development began after the end of the Cultural Revolution and is still ongoing. During the recovery stage, some works on Neo-Confucianism re-emerged, prompting academia to reflect on and discuss the merits and demerits of Neo-Confucianism. The development stage is characterized by a comprehensive and thorough reevaluation of Confucianism and Neo-Confucianism, drawing on the essence of ancient wisdom to address contemporary challenges.

  \bibliography{book.bib,packages.bib}

\end{document}
